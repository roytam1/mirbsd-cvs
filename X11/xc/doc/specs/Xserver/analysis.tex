\documentstyle{article}
\setlength{\parindent}{0 pt}
\setlength{\parskip}{6pt}

% $XFree86: xc/doc/specs/Xserver/analysis.tex,v 1.2 2001/04/19 23:36:35 dawes Exp $

\begin{document}

\title{Analysis of the X Protocol for Security Concerns\\Draft Version 2}
\author{David P. Wiggins\\X Consortium, Inc.}
\date{May 10, 1996}
\maketitle

\begin{abstract}

This paper attempts to list all instances of certain types of security
problems in the X Protocol.  Issues with authorization are not
addressed.  We assume that a malicious client has already succeeded in
connecting, and try to assess what harm it can then do.  We propose
modifications to the semantics of the X Protocol to reduce these
risks.

\end{abstract}
% suppress page number on title page
\thispagestyle{empty}

\eject

Copyright \copyright 1996 X Consortium, Inc.  All Rights Reserved.

THE SOFTWARE IS PROVIDED "AS IS", WITHOUT WARRANTY OF ANY KIND,
EXPRESS OR IMPLIED, INCLUDING BUT NOT LIMITED TO THE WARRANTIES OF
MERCHANTABILITY, FITNESS FOR A PARTICULAR PURPOSE AND NONINFRINGEMENT.
IN NO EVENT SHALL THE X CONSORTIUM BE LIABLE FOR ANY CLAIM, DAMAGES OR
OTHER LIABILITY, WHETHER IN AN ACTION OF CONTRACT, TORT OR OTHERWISE,
ARISING FROM, OUT OF OR IN CONNECTION WITH THE SOFTWARE OR THE USE OF
OR OTHER DEALINGS IN THE SOFTWARE.

Except as contained in this notice, the name of the X Consortium shall
not be used in advertising or otherwise to promote the sale, use or
other dealings in this Software without prior written authorization
from the X Consortium.

\eject

\section{Definition of Threats}

We analyze the X protocol for the following threats.

\begin{description}

\item[Theft] occurs when a client gains access to information owned by
another client without explicit permission from that other client.
For this analysis, we take a broad view of ownership: any information
that exists in the server due to the actions of a client is considered
owned by that client.  Furthermore, the client that has input focus
owns keyboard events, and the client that owns the window that the
pointer is in owns mouse events.  This view may reveal certain
instances of ``theft'' that we don't care to stop, but we think it is
better to identify all potential candidates up front and cull the list
later than to do a partial analysis now and plan on reanalyzing for
remaining holes later.

\item[Denial of service] occurs when a client causes another client or
the user to lose the ability to perform some operation.

\item[Spoofing] occurs when a client attempts to mimic another client
with the hope that the user will interact with it as if it really were
the mimicked client.  A wide variety of requests may be used in a
spoofing attack; we will only point out a few likely candidates.

\item[Destruction] occurs when a client causes another client to lose
information in a way that the client or user is likely to notice.
(This does not count expected forms of destruction, e.g., exposures.)

\item[Alteration] occurs when a client causes another client to lose
information in a way that the client or user is unlikely to
notice. e.g., changing one pixel in a drawable.

\end{description}

The line between alteration and destruction is subjective.  Security
literature does often distinguish between them, though not always
explicitly.  Alteration is often considered more insidious because its
effects may not be realized until long after it has occurred.  In the
intervening time, each time the altered data is used, it can cause
more damage.



\section{General security concerns and remedies}

The following sections discuss security problems intrinsic to the X
Protocol.  A statement of each problem is usually followed by
potential remedies.  A few words here about possible remedies will
help frame the specific ones described below.

If a client attempts a threatening operation, the server may take one
of the following actions, listed roughly in order of severity:

1. Execute the request normally.  This is the right choice when we
decide that a particlar threat is not serious enough to worry about.

2. Execute the request in some modified form, e.g., substitute
different values for some of the request fields, or edit the reply.

3. Arrange to ask the user what to do, given some subset of the other
choices in this list.  This must be used sparingly because of the
performance impact.

4. Treat the request as a no-op.  If the client will probably not
notice, or if it seems likely that the intent was benign, this is a
good choice.
 
5. Send a protocol error to the client.  If the client will be
confused enough by the other options that it will probably crash or
become useless anyway, or if it seems likely that the intent was
malicious, this is a good choice.

6. Kill the client.  This might be the right action if there is no
doubt that the client is hostile.

In most cases we present the one option that seems most appropriate to
counter the threat, taking into account the seriousness of the threat,
the implementation difficulty, and the impact on applications.  Our
initial bias is to err on the side of stronger security, with the
accompanying tighter restrictions.  As we uncover important operations
and applications that the new restrictions interfere with, we can apply
selective loosening to allow the desired functionality.

In some cases we will suggest returning an Access error where the X
protocol does not explicitly allow one.  These new Access errors arise
when a client can only perform a (non-empty) subset of the defined
operations on a resource.  The disallowed operations cause Access
errors.  The resiource at issue is usually a root window.



\subsection{Access to Server Resources}

The X protocol allows clients to manipulate resources (objects)
belonging to other clients or to the server.  Any request that
specifies a resource ID is vulnerable to some of the above threats.
Such requests also provide a way for a client to guess resource IDs of
other clients.  A client can make educated guesses for possible
resource IDs, and if the request succeeds, it knows it has determined
a valid resource ID.  We call this ``resource ID guessing'' in the
list below.

One likely defense against these problems is to have the server send
an appropriate protocol error to deny the existence of any resource
specified by a client that doesn't belong to that client.  A variation
on this policy lets cooperating groups of clients access each other's
resources, but not those of other groups.  The Broadway project will
initially use a less general form of this idea by having two groups,
trusted and untrusted.  Trusted clients can do everything that X
clients can do today.  They will be protected from untrusted clients
in ways described below.  Untrusted clients will not be protected from
each other.  Though this will be the initial design, we need to make
sure there is a growth path to multiple (more than two) groups.

Most of the time, applications never access server resources that
aren't their own, so the impact of disallowing such accesses should be
minimal.  There are a few notable exceptions, most of which will be
discussed under the relevant protocol requests.  They are: ICCCM
selection transfer, Motif drag and drop, and server-global resources
like the root window and default colormap.  Another major exception is
the window manager, which routinely manipulates windows of other
applications.  The solution for window managers is to always run them
as trusted applications.

The implementation difficulty of limiting access to resources should
not be large.  All resource accesses eventually funnel down to one of
two functions in dix/resource.c: LookupIDByType() and
LookupIDByClass().  A few lines of checking at the top of these
functions will form the heart of this defense.  There is a small
problem because these functions are not told which client is doing the
lookup, but that can be solved either by adding a client parameter
(probably as a new function to preserve compatibility), or by using
the server global requestingClient.

ISSUE: are we really going to be able to get away with hiding trusted
resources, or will things like Motif drag and drop force us to expose
them?  (Either way, the operations that untrusted clients can do to
trusted resources will have to be limited.)  Is there something in Xt
or the ICCCM that breaks if you hide resources?

\subsection{Denial of Service}

\subsubsection{Memory Exhaustion}

Any request that causes the server to consume resources (particularly
memory) can be used in a denial of service attack.  A client can use
such requests repeatedly until the server runs out of memory.  When
that happens, the server will either crash or be forced to send Alloc
errors.  The most obvious candidates are resource creation requests,
e.g., CreatePixmap, but in reality a large percentage of requests
cause memory allocation, if only temporarily, depending on the server
implementation.  For this reason, the list of requests subject to this
form of denial of service will be necessarily incomplete.

To address this form of denial of service, the server could set
per-client quotas on memory consumption.  When the limit is surpassed,
the server could return Alloc errors.  The application impact is
minimal as long as the application stays within quota.  The
implementation difficulty is another story.

Conceptually, it seems easy: simply have a way to set the limit, and
on every memory (de)allocation operation, update the client's current
usage, and return an error if the client is over the limit.  The first
problem is something we've already touched on: the allocator functions
aren't told which client the allocation belongs to.  Unlike resource
lookups, allocations are done in too many places to consider a new
interface that passes the client, so using the global requestingClient
is practically mandatory.

The problems run deeper.  The logical thing for the allocator to do if
the client is over its limit is to return NULL, indicating allocation
failure.  Unfortunately, there are many places in the server that will
react badly if this happens.  Most of these places, but not all, are
``protected'' by setting the global variable Must\_have\_memory to True
around the delicate code.  We could help the problem by skipping the
limit check if Must\_have\_memory is True.  The best solution would be
to bullet-proof the server against allocation failures, but that is
beyond the scope of Broadway.  Another consideration is that the
additional checking may have a measurable performance impact, since
the server does frequent allocations.

A third problem is that there is no portable way to determine the size
of a chunk of allocated memory given just a pointer to the chunk, and
that's all you have inside Xrealloc() and Xfree().  The server could
compensate by recording the sizes itself somewhere, but that would be
wasteful of memory, since the malloc implementation also must be
recording block sizes.  On top of that, the redundant bookkeeping
would hurt performance.  One solution is to use a custom malloc that
has the needed support, but that too seems beyond the scope of
Broadway.

Considering all of this, we think it is advisable to defer solving the
memory exhaustion problem to a future release.  Keep this in mind when
you see quotas mentioned as a defense in the list below.

\subsubsection{CPU Monopolization}

Another general way that a client can cause denial of service is to
flood the server with requests.  The server will spend a large
percentage of its time servicing those requests, possibly starving
other clients and certainly hurting performance.  Every request can be
used for flooding, so we will not bother to list flooding on every
request.  A variation on this attack is to flood the server with new
connection attempts.

To reduce the effectiveness of flooding, the server could use a
different scheduling algorithm that throttles clients that are
monopolizing the server, or it could simply favor trusted clients over
untrusted ones.  Applications cannot depend on a particular scheduling
algorithm anyway, so changing it should not affect them.  The
Synchronization extension specifies a way to set client priorities,
and a simple priority scheduler already exists in the server to
support it, so this should be simple to add.



\section{Security concerns with specific window attributes}

\subsection{Background-pixmap}

Clients can use windows with the background-pixmap attribute set to
None (hereafter ``background none windows'') to obtain images of other
windows.  A background none window never paints its own background, so
whatever happened to be on the screen when the window was mapped can
be read from the background none window with GetImage.  This may well
contain data from other windows.  The CreateWindow and
ChangeWindowAttributes requests can set the background-pixmap
attribute set to None, and many window operations can cause data from
other windows to be left in a background none window, including
ReparentWindow, MapWindow, MapSubwindows, ConfigureWindow, and
CirculateWindow.

Background none windows can also be used to cause apparent alteration.
A client can create a window with background none and draw to it.  The
drawing will appear to the user to be in the windows below the
background none window.

To remedy these problems, the server could substitute a well-defined
background when a client specifies None.  Ideally the substituted
background would look different enough from other windows that the
user wouldn't be confused.  A tile depicting some appropriate
international symbol might be reasonable.  We believe that there are
few applications that actually rely on background none semantics, and
those that do will be easy for the user to identify because of the
distinctive tile.  Implementation should not be a problem either.
Luckily, the window background cannot be retrieved through the X
protocol, so we won't have to maintain any illusions about its value.

ISSUE: Some vendors have extensions to let you query the window
background.  Do we need to accomodate that?

ISSUE: Will this lead to unacceptable application breakage?  Could the
server be smarter, only painting with the well-defined background when
the window actually contains bits from trusted windows?

\subsection{ParentRelative and CopyFromParent}

Several window attributes can take on special values that cause them
to reference (ParentRelative) or copy (CopyFromParent) the same
attribute from the window's parent.  This fits our definition of theft.
The window attributes are class, background-pixmap, border-pixmap, and
colormap.  All of these can be set with CreateWindow; all but class
can be set with ChangeWindowAttributes.

These forms of theft aren't particularly serious, so sending an error
doesn't seem appropriate.  Substitution of different attribute values
seems to be the only reasonable option, and even that is likely to
cause trouble for clients.  Untrusted clients are already going to be
prevented from creating windows that are children of trusted clients
(see CreateWindow below).  We recommend that nothing more be done to
counter this threat.


\subsection{Override-redirect}

Windows with the override-redirect bit set to True are generally
ignored by the window manager.  A client can map an override-redirect
window that covers most or all of the screen, causing denial of
service since other applications won't be visible.

To prevent this, the server could prevent more than a certain
percentage (configurable) the of screen area from being covered by
override-redirect windows of untrusted clients.

Override-redirect windows also make some spoofing attacks easier since
the client can more carefully control the presentation of the window
to mimic another client.  Defenses against spoofing will be
given under MapWindow.

\section{Security concerns with specific requests}

To reduce the space needed to discuss 120 requests, most of the
following sections use a stylized format.  A threat is given, followed
by an imperative statement.  The implied subject is an untrusted
client, and the object is usually a trusted client.  Following that,
another statement starting with ``Defense:'' recommends a
countermeasure for the preceding threat(s).

Resources owned by the server, such as the root window and the default
colormap, are considered to be owned by a trusted client.


\subsection{CreateWindow}

Alteration: create a window as a child of another client's window,
altering its list of children.

Defense: send Window error.  Specifying the root window as the parent will
have to be allowed, though.

Theft: create an InputOnly window or a window with background none on
top of other clients' windows, select for keyboard/mouse input on that
window, and steal the input.  The input can be resent using SendEvent
or an input synthesis extension so that the snooped application
continues to function, though this won't work convincingly with the
background none case because the drawing will be clipped.

Defense: send an error if a top-level InputOnly window is created (or
reparented to the root).  Countermeasures for background none and
SendEvent are discussed elsewhere.

ISSUE: The Motif drag and drop protocol creates and maps such a window
(at $-$100,$-$100, size 10x10) to ``cache frequently needed data on
window properties to reduce roundtrip server requests.''  Proposed
solution: we could only send an error if the window is visible, which
would require checking in, MapWindow, ConfigureWindow, and
ReparentWindow.

Theft: resource ID guessing (parent, background-pixmap, border-pixmap,
colormap, and cursor).

Defense: send Window, Pixmap, Colormap, or Cursor error.

Denial of service: create windows until the server runs out of memory.

Defense: quotas.

Also see section 3.


\subsection{ChangeWindowAttributes}

Alteration: change the attributes of another client's window.

Theft: select for events on another client's window.

Defense for both of the above: send Window error.

ISSUE: The Motif drop protocol states that ``the initiator should
select for DestroyNotify on the destination window such that it is
aware of a potential receiver crash.''  This will be a problem if the
initiator is an untrusted window and the destination is trusted.  Can
the server, perhaps with the help of the security manager, recognize
that a drop is in progress and allow the DestroyNotify event selection
in this limited case?

ISSUE: The Motif pre-register drag protocol probably requires the
initiator to select for Enter/LeaveNotify on all top-level windows.
Same problem as the previous issue.

Theft: resource ID guessing (background-pixmap, border-pixmap,
colormap, and cursor).

Defense: send Pixmap, Colormap, or Cursor error.

Also see section 3.


\subsection{GetWindowAttributes}

Theft: get the attributes of another client's window.

Theft: resource ID guessing (window).

Defense for both of the above: send Window error.


\subsection{DestroyWindow, DestroySubwindows}

Destruction: destroy another client's window.

Theft: resource ID guessing (window).

Defense for both of the above: send Window error.


\subsection{ChangeSaveSet}

Alteration: cause another client's windows to be reparented to the
root when this client disconnects (only if the other client's windows
are subwindows of this client's windows).

Defense: process the request normally.  The trusted client gives away
some of its protection by creating a subwindow of an untrusted window.

Theft: resource ID guessing (window).

Defense: send Window error.


\subsection{MapWindow}

Spoofing: map a window that is designed to resemble a window of
another client.  Additional requests will probably be needed to
complete the illusion.

Defense:

We consider spoofing to be a significant danger only if the user is
convinced to interact with the spoof window.  The defense centers on
providing enough information to enable the user to know where
keyboard, mouse, and extension device input is going.  To accomplish
this, the server will cooperate with the security manager, an external
process.  The server will provide the following facilities to the
security manager:

1.  A way to create a single window that is unobscurable by any window
of any other client, trusted or untrusted.  It needs to be
unobscurable so that it is spoof-proof.

ISSUE: is a weaker form of unobscurability better?  Should the window be
obscurable by trusted windows, for example?

ISSUE: does unobscurable mean that it is a child of the root that is
always on top in the stacking order?

2.  A way to determine if a given window ID belongs to an untrusted
client.

The security manager will need to select for the existing events
FocusIn, FocusOut, EnterNotify, LeaveNotify, DeviceFocusIn, and
DeviceFocusOut on all windows to track what window(s) the user's input
is going to.  Using the above server facilities, it can reliably
display the trusted/untrusted status of all clients currently
receiving input.

ISSUE: is it too much to ask the security manager to select for all
these events on every window?  Do we need to provide new events that
you select for *on the device* that tell where the device is focused?

None of this should have any application impact.

The unobscurable window may be tricky to implement.  There is already
some machinery in the server to make an unobscurable window for the
screen saver, which may help but may also get in the way now that we
have to deal with two unobscurable windows.



\subsection{Window Operations}

Specifically, ReparentWindow, MapWindow, MapSubwindows, UnmapWindow,
UnmapSubwindows, ConfigureWindow, and CirculateWindow.

Alteration: manipulate another client's window.

Theft: resource ID guessing (window, sibling).

Defense for both of the above: send a Window error unless it is a root
window, in which case we should send an Access error.



\subsection{GetGeometry}

Theft: get the geometry of another client's drawable.

Theft: resource ID guessing (drawable).

Defense for both of the above: send Drawable error.  However, root
windows will be allowed.



\subsection{QueryTree}

Theft: resource ID guessing (window).

Defense: send Window error.

Theft: discover window IDs that belong to other clients.

Defense: For the child windows, censor the reply by removing window
IDs that belong to trusted clients.  Allow the root window to be
returned.  For the parent window, if it belongs to a trusted client,
return the closest ancestor window that belongs to an untrusted
client, or if such a window does not exist, return the root window for
the parent window.

ISSUE: will some applications be confused if we filter out the window
manager frame window(s), or other windows between the queried window
and the root window?

ISSUE: the Motif drag protocol (both preregister and dynamic) needs to
be able to locate other top-level windows for potential drop sites.
See also section 2.1.


\subsection{InternAtom}

Theft: discover atom values of atoms interned by other clients.
This lets you determine if a specific set of atoms has been
interned, which may lead to other inferences.

Defense: This is a minor form of theft.  Blocking it will interfere
with many types of inter-client communication.  We propose to do
nothing about this threat.

Denial of service: intern atoms until the server runs out of memory.

Defense: quotas.



\subsection{GetAtomName}

Theft: discover atom names of atoms interned by other clients.
This lets you determine if a specific set of atoms has been
interned, which may lead to other inferences.

Defense: This is a minor form of theft.  We propose to do nothing
about this threat.



\subsection{ChangeProperty}

Alteration: change a property on another client's window or one that
was stored by another client.

Theft: resource ID guessing (window).

Defense for both of the above: send Window error.

ISSUE: Selection transfer requires the selection owner to change a
property on the requestor's window.  Does the security manager get us
out of this?  Does the server notice the property name and window
passed in ConvertSelection and temporarily allow that window property
to be written?

ISSUE: should certain root window properties be writable?

Denial of service: store additional property data until the server
runs out of memory.

Defense: quotas.



\subsection{DeleteProperty}

Destruction: delete a property stored by another client.

Theft: resource ID guessing (window).

Defense for both of the above: send Window error.



\subsection{GetProperty}

Theft: get a property stored by another client.

Theft: resource ID guessing (window).

Defense for both of the above: send Window error.

ISSUE: should certain root window properties be readable?  Proposed
answer: yes, some configurable list.  Do those properties need to be
polyinstantiated?

ISSUE: Motif drag and drop needs to be able to read the following
properties: WM\_STATE to identify top-level windows, \_MOTIF\_DRAG\_WINDOW
on the root window, \_MOTIF\_DRAG\_TARGETS on the window given in the
\_MOTIF\_DRAG\_WINDOW property, and \_MOTIF\_DRAG\_RECEIVER\_INFO on windows
with drop sites.  Additionally, some properties are needed that do not
have fixed names.


\subsection{RotateProperties}

Alteration: rotate properties stored by another client.

Theft: resource ID guessing (window).

Defense for both of the above: send Window error.



\subsection{ListProperties}

Theft: list properties stored by another client.

Theft: resource ID guessing (window).

Defense for both of the above: send Window error.

ISSUE: should certain root window properties be listable?



\subsection{SetSelectionOwner}

Theft: Steal ownership of a selection.

Denial of service: do this repeatedly so that no other client can own
the selection.

Defense for both of the above: have a configurable list of selections
that untrusted clients can own.  For other selections, treat this
request as a no-op.

ISSUE: how does the security manager get involved here?  Is it the one
that has the configurable list of selections instead of the server?

Theft: resource ID guessing (window).

Defense: send Window error.



\subsection{GetSelectionOwner}

Theft: discover the ID of another client's window via the owner field
of the reply.

Defense: if the selection is on the configurable list mentioned above,
return the root window ID, else return None.

ISSUE: how does the security manager get involved here?



\subsection{ConvertSelection}

Theft: this initiates a selection transfer (see the ICCCM) which sends
the selection contents from the selection owner, which may be another
client, to the requesting client.

Defense: since in many cases ConvertSelection is done in direct
response to user interaction, it is probably best not to force it to
fail, either silently or with an error.  The server should rely on the
security manager to assist in handling the selection transfer.

Theft: resource ID guessing (requestor).

Defense: send Window error.



\subsection{SendEvent}

A client can use SendEvent to cause events of any type to be sent to
windows of other clients.  Similarly, a client could SendEvent to one
of its own windows with propogate set to True and arrange for the
event to be propogated up to a window it does not own.  Clients can
detect events generated by SendEvent, but we cannot assume that they
will.

Defense: ignore this request unless the event being sent is a
ClientMessage event, which should be sent normally so that selection
transfer, Motif drag and drop, and certain input methods have a chance
at working.

ISSUE: does allowing all ClientMessages open up too big a hole?

Theft: resource ID guessing (window).

Defense: send Window error.



\subsection{Keyboard and Pointer Grabs}

Specifically, GrabKeyboard, GrabPointer, GrabKey, and GrabButton.

Denial of service/Theft: take over the keyboard and pointer.  This
could be viewed as denial of service since it prevents other clients
from getting keyboard or mouse input, or it could be viewed as theft
since the user input may not have been intended for the grabbing
client.

Defense: provide a way to break grabs via some keystroke combination,
and have a status area that shows which client is getting input.
(See MapWindow.)

Theft: resource ID guessing (grab-window, confine-to, cursor).

Defense: send Window or Cursor error.



\subsection{ChangeActivePointerGrab}

Theft: resource ID guessing (cursor).

Defense: send Cursor error.



\subsection{GrabServer}

Denial of service: a client can grab the server and not let go,
locking out all other clients.

Defense: provide a way to break grabs via some keystroke combination.



\subsection{QueryPointer}

Theft: A client can steal pointer motion and position, button input,
modifier key state, and possibly a window of another client with this
request.

Defense: if the querying client doesn't have the pointer grabbed, and
the pointer is not in one of its windows, the information can be
zeroed.

Theft: resource ID guessing (window).

Defense: send Window error.



\subsection{GetMotionEvents}

Theft: steal pointer motion input that went to other clients.

Defense: ideally, the server would return only pointer input that was
not delivered to any trusted client.  The implementation effort to do
that probably outweighs the marginal benefits.  Instead, we will
always return an empty list of motion events to untrusted clients.

Theft: resource ID guessing (window).

Defense: send Window error.



\subsection{TranslateCoordinates}

Theft: discover information about other clients' windows: position,
screen, and possibly the ID of one of their subwindows.

Defense: send an error if src-window or dst-window do not belong
to the requesting client.

Theft: resource ID guessing (src-window, dst-window).

Defense: send Window error.



\subsection{WarpPointer}

A client can cause pointer motion to occur in another client's window.

Denial of service: repeated pointer warping prevents the user from
using the mouse normally.

Defense for both of the above: if the querying client doesn't have the
pointer grabbed, and the pointer is not in one of its windows, treat
the request as a no-op.

Theft: resource ID guessing (src-window, dst-window).

Defense: send Window error.



\subsection{SetInputFocus}

Theft: a client can use this request to make one of its own windows
have the input focus (keyboard focus).  The user may be unaware that
keystrokes are now going to a different window.

Denial of service: repeatedly setting input focus prevents normal use
of the keyboard.

Defense for both of the above: only allow untrusted clients to
SetInputFocus if input focus is currently held by another untrusted
client.

ISSUE: this will break clients using the Globally Active Input model
described in section 4.1.7 of the ICCCM.

Theft: resource ID guessing (focus).

Defense: send Window error.



\subsection{GetInputFocus}

Theft: the reply may contain the ID of another client's window.

Defense: return a focus window of None if a trusted client currently
has the input focus.



\subsection{QueryKeymap}

Theft: poll the keyboard with this to see which keys are being pressed.

Defense: zero the returned bit vector if a trusted client currently
has the input focus.



\subsection{Font Requests}

Specifically, OpenFont, QueryFont, ListFonts, ListFontsWithInfo, and
QueryTextExtents.

Theft: discover font name, glyph, and metric information about fonts
that were provided by another client (by setting the font path).
Whether it is theft to retrieve information about fonts from the
server's initial font path depends on whether or not you believe those
fonts, by their existence in the initial font path, are intended to be
globally accessible by all clients.

Defense:

Maintain two separate font paths, one for trusted clients and one for
untrusted clients.  They are both initialized to the default font path
at server reset.  Subsequently, changes to one do not affect the
other.  Since untrusted clients will not see font path elements added
by trusted clients, they will not be able to access any fonts provided
by those font path elements.

Theft: resource ID guessing (font) (QueryFont and QueryTextExtents only).

Defense: send Font error.

Denial of service: open fonts until the server runs out of memory
(OpenFont only).

Defense: quotas.


\subsection{CloseFont}

Destruction: close another client's font.

Defense: send Font error.



\subsection{SetFontPath}

Denial of service: change the font path so that other clients cannot
find their fonts.

Alteration: change the font path so that other clients get different
fonts than they expected.

Defense for both of the above: separate font paths for trusted and
untrusted clients, as described in the Font Requests section.

ISSUE: the printing project considered per-client font paths and
concluded that it was very difficult to do.  We should look at this
aspect of the print server design to see if we can reuse the same
scheme.  We should also try to reconstruct what was so difficult about
this; it doesn't seem that hard on the surface.



\subsection{GetFontPath}

Theft: retrieve font path elements that were set by other clients.

Use knowledge from font path elements to mount other attacks,
e.g., attack a font server found in the font path.

Defense for both of the above: separate font paths for trusted and
untrusted clients, as described in the Font Requests section.



\subsection{CreatePixmap}

Theft: resource ID guessing (drawable).

Defense: send Drawable error.

Denial of service: create pixmaps until the server runs out of memory.

Defense: quotas.



\subsection{FreePixmap}

Destruction: destroy another client's pixmap.

Defense: send Pixmap error.


\subsection{CreateGC}

Theft: resource ID guessing (drawable, tile, stipple, font, clip-mask).

Defense: send Drawable, Pixmap, or Font error.

Denial of service: create GCs until the server runs out of memory.

Defense: quotas.



\subsection{CopyGC}

Theft: copy GC values of another client's GC.

Alteration: copy GC values to another client's GC.

Defense for both of the above: send GC error.



\subsection{ChangeGC, SetDashes, SetClipRectangles}

Alteration: change values of another client's GC.

Theft: resource ID guessing (gc, tile, stipple, font, clip-mask)
(last four for ChangeGC only).

Defense for both of the above: send GC error.



\subsection{FreeGC}

Destruction: destroy another client's GC.

Defense: send GC error.



\subsection{Drawing Requests}

Specifically, ClearArea, CopyArea, CopyPlane, PolyPoint,
PolyLine, PolySegment, PolyRectangle, PolyArc, FillPoly,
PolyFillRectangle, PolyFillArc, PutImage, PolyText8, PolyText16,
ImageText8, and ImageText16.

Alteration: draw to another client's drawable.

Theft: resource ID guessing:
	ClearArea - window;
	CopyArea, CopyPlane - src-drawable, dst-drawable, gc;
	all others - drawable, gc.

Defense for both of the above: send appropriate error.

ISSUE: The Motif preregister drag protocol requires clients to draw
into windows of other clients for drag-over/under effects.

Spoofing: draw to a window to make it resemble a window of
another client.

Defense: see MapWindow.



\subsection{GetImage}

Theft: get the image of another client's drawable.

Theft: resource ID guessing (drawable).

Defense: send Drawable error.

Theft: get the image of your own window, which may contain pieces of
other overlapping windows.

Defense: censor returned images by blotting out areas that contain
data from trusted windows.



\subsection{CreateColormap}

Theft: resource ID guessing (window).

Defense: send Colormap error.

Denial of service: create colormaps with this request until the server
runs out of memory.

Defense: quotas.



\subsection{FreeColormap}

Destruction: destroy another client's colormap.

Defense: send Colormap error.



\subsection{CopyColormapAndFree}

Theft: resource ID guessing (src-map).

Defense: send Colormap error.  However, default colormaps will be
allowed.

ISSUE: must untrusted applications be allowed to use standard colormaps?
(Same issue for ListInstalledColormaps, Color Allocation Requests,
FreeColors, StoreColors, StoreNamedColor, QueryColors, and LookupColor.)

Denial of service: create colormaps with this request until the server
runs out of memory.

Defense: quotas.



\subsection{InstallColormap, UninstallColormap}

Theft: resource ID guessing.

Defense: send Colormap error.

Denial of service: (un)install any colormap, potentially preventing
windows from displaying correct colors.

Defense: treat this request as a no-op.  Section 4.1.8 of the ICCCM
states that (un)installing colormaps is the responsibility of the window
manager alone.

ISSUE: the ICCCM also allows clients to do colormap installs if the
client has the pointer grabbed.  Do we need to allow that too?



\subsection{ListInstalledColormaps}

Theft: resource ID guessing (window).

Defense: send Colormap error.

Theft: discover the resource ID of another client's colormap
from the reply.

Defense: remove the returned colormap IDs; only let through default
colormaps and colormaps of untrusted clients.



\subsection{Color Allocation Requests}

Specifically, AllocColor, AllocNamedColor, AllocColorCells, and
AllocColorPlanes.

Alteration/Denial of service: allocate colors in another client's
colormap.  It is denial of service if the owning client's color
allocations fail because there are no cells available.  Otherwise it
is just alteration.

Theft: resource ID guessing (cmap).

Defense for both of the above: send Colormap error.  However, default
colormaps will be allowed.



\subsection{FreeColors}

Theft: resource ID guessing (cmap).

Defense: send Colormap error.  However, default colormaps will be
allowed.



\subsection{StoreColors, StoreNamedColor}

Alteration: change the colors in another client's colormap.

Theft: resource ID guessing (cmap).

Defense for both of the above: send Colormap error.  However, default
colormaps will be allowed.



\subsection{QueryColors, LookupColor}

Theft: retrieve information about the colors in another client's
colormap.

Theft: resource ID guessing (cmap).

Defense for both of the above: send Colormap error.  However, default
colormaps will be allowed.



\subsection{CreateCursor, CreateGlyphCursor}

Theft: resource ID guessing (source, mask or source-font, mask-font).

Defense: send Pixmap or Font error.  However, the default font will be
allowed.

Denial of service: create cursors until the server runs out of memory.

Defense: quotas.



\subsection{FreeCursor}

Destruction: free another client's cursor.

Defense: send Cursor error.



\subsection{RecolorCursor}

Alteration: recolor another client's cursor.

Theft: resource ID guessing (cursor).

Defense for both of the above: send Cursor error.



\subsection{QueryBestSize}

Theft: resource ID guessing (drawable).

Defense: send Drawable error.



\subsection{ListExtensions, QueryExtension}

Determine the extensions supported by the server, and use the list to
choose extension-specific attacks to attempt.

Defense: extensions will have a way to tell the server whether it is
safe for untrusted clients to use them.  These requests will only
return information about extensions that claim to be safe.



\subsection{Keyboard configuration requests}

Specifically, ChangeKeyboardControl, ChangeKeyboardMapping,
and SetModifierMapping.

Alteration: change the keyboard parameters that were established by
another client.

Denial of service: with ChangeKeyboardControl, disable auto-repeat,
key click, or the bell.  With ChangeKeyboardMapping or
SetModifierMapping, change the key mappings so that the keyboard is
difficult or impossible to use.

Defense for both of the above: treat these requests as a no-op.



\subsection{Keyboard query requests}

Specifically, GetKeyboardControl, GetKeyboardMapping, and
GetModifierMapping.

Theft: get keyboard information that was established by another
client.

Defense: This is a minor form of theft.  We propose to do nothing
about this threat.



\subsection{ChangePointerControl, SetPointerMapping}

Alteration: change the pointer parameters that were established by
another client.

Denial of service: set the pointer parameters so that the pointer is
difficult or impossible to use.

Defense for both of the above: treat these requests as a no-op.



\subsection{GetPointerControl, GetPointerMapping}

Theft: get pointer parameters that were established by another client.

Defense: This is a minor form of theft.  We propose to do nothing
about this threat.



\subsection{SetScreenSaver}

Alteration: change the screen saver parameters that were established
by another client.

Denial of service: set the screen saver parameters so that the screen
saver is always on or always off.

Defense for both of the above: treat these requests as a no-op.



\subsection{GetScreenSaver}

Theft: get screen saver parameters that were established by another
client.

Defense: This is a minor form of theft.  We propose to do nothing
about this threat.



\subsection{ForceScreenSaver}

Denial of service: repeatedly activate the screen saver so that the
user cannot see the screen as it would look when the screen saver
is off.

Denial of service: repeatedly reset the screen saver, preventing it
from activating.

Defense for both of the above: treat these requests as a no-op.



\subsection{ChangeHost}

Most servers already have some restrictions on which clients can use
this request, so whether the following list applies is implementation
dependent.

Denial of service: remove a host from the list, preventing clients
from connecting from that host.

Add a host to the list.  Clients from that host may then launch
other attacks of any type.

Defense for both of the above: return Access error.


\subsection{ListHosts}

Theft: steal host identities and possibly even user identities that
are allowed to connect.

Launch attacks of any type against the stolen host/user identities.

Defense for both of the above: return only untrusted hosts.



\subsection{SetAccessControl}

Most servers already have some restrictions on which clients can use
this request, so whether the following list applies is implementation
dependent.

Alteration: change the access control value established by some other
client.

Disable access control, allowing clients to connect who would normally
not be able to connect.  Those clients may then launch other attacks
of any type.

Defense for both of the above: return Access error.



\subsection{SetCloseDownMode}

Denial of service: set the close-down mode to RetainPermanent or
RetainTemporary, then disconnect.  The server cannot reuse the
resource-id-base of the disconnected client, or the memory used by the
retained resources, unless another client issues an appropriate
KillClient to cancel the retainment.  The server has a limited number
of resource-id-bases, and when they are exhausted, it will be unable
to accept new client connections.

Defense: treat this request as a no-op.


\subsection{KillClient}

Destruction/Denial of service: kill another currently connected
client.

Destruction: kill a client that has terminated with close-down mode
of RetainTemporary or RetainPermanent, destroying all its retained
resources.

Destruction: specify AllTemporary as the resource, destroying all
resources of clients that have terminated with close-down mode
RetainTemporary.

Defense for all of the above: return Value error.



\subsection{Clean Requests}

Other than denial of service caused by flooding, these requests have
no known security concerns: AllowEvents, UngrabPointer, UngrabButton,
UngrabKeyboard, UngrabKey, UngrabServer, NoOperation, and Bell.



\section{Events}

The only threat posed by events is theft.  Selecting for events on
another client's resources is always theft.  We restrict further
analysis by assuming that the client only selects for events on its
own resources, then asking whether the events provide information
about other clients.



\subsection{KeymapNotify}

Theft: the state of the keyboard can be seen when the client does not
have the input focus.  This is possible because a KeymapNotify is
sent to a window after every EnterNotify even if the window does not
have input focus.

Defense: zero the returned bit vector if a trusted client currently
has the input focus.



\subsection{Expose}

Theft: discover where other clients' windows overlap your own.  For
instance, map a full-screen window, lower it, then raise it.  The
resulting exposes tell you where other windows are.

Defense: about the only thing you could do here is force backing store
to be used on untrusted windows, but that would probably use too much
server memory.  We propose to do nothing about this threat.



\subsection{GraphicsExposure}

Theft: discover where other clients' windows overlap your own.
For instance, use CopyArea to copy the entire window's area exactly
on top of itself.  The resulting GraphicsExposures tell you where
the window was obscured.

Defense: see Expose above.  We propose to do nothing about this
threat.


\subsection{VisibilityNotify}

Theft: this event provides crude positional information about other
clients, though the receiver cannot tell which other clients.

Defense: The information content of this event is very low.  We
propose to do nothing about this threat.



\subsection{ReparentNotify}

Theft: the parent window may belong to some other client (probably
the window manager).

Defense: If the parent window belongs to a trusted client, return the
closest ancestor window that belongs to an untrusted client, or if
such a window does not exist, return the root window for the parent
window.

ISSUE: what is the application impact?


\subsection{ConfigureNotify}

Theft: the above-sibling window may belong to some other client.

Defense: return None for the above-sibling window if it belongs to a
trusted client.

ISSUE: what is the application impact?


\subsection{ConfigureRequest}

Theft: the sibling window may belong to some other client.

Defense: return None for the sibling window if it belongs to a trusted
client.

ISSUE: what is the application impact?


\subsection{SelectionClear}

Theft: the owner window may belong to some other client.

Defense: return None for the owner window if it belongs to a trusted
client.



\subsection{SelectionRequest}

Theft: the requestor window may belong to some other client.

Defense: Blocking this event or censoring the window would prevent
selection transfers from untrusted clients to trusted clients from
working.  We propose to do nothing in the server about this threat.
The security manager may reduce the exposure of trusted window IDs
by becoming the owner of all selections.



\subsection{MappingNotify}

Theft: discover keyboard, pointer, or modifier mapping information
set by another client.

Defense: Any tampering with this event will cause clients to have an
inconsistent view of the keyboard or pointer button configuration,
which is likely to confuse the user.  We propose to do nothing about
this threat.


\section{Errors}

There appear to be no threats related to procotol errors.



\section{Future Work}

The next steps are resolve the items marked ISSUE and to decide if the
defenses proposed are reasonable.  Discussion on the security@x.org
mailing list, prototyping, and/or starting the implementation should
help answer these questions.



\section{References}

Bellcore, ``Framework Generic Requirements for X Window System Security,''
Technical Advisory FA-STS-001324, Issue 1, August 1992.

Dardailler, Daniel, ``Motif Drag And Drop Protocol,'' unpublished
design notes.

Kahn, Brian L., ``Safe Use of X WINDOW SYSTEM protocol Across a
Firewall'', unpublished draft, The MITRE Corporation, 1995.

Rosenthal, David S. H., ``LINX - a Less INsecure X server,'' Sun Microsystems,
29th April 1989.

Rosenthal, David and Marks, Stuart W., ``Inter-Client Communication
Conventions Manual Version 2.0,''
{\tt ftp://ftp.x.org/pub/R6.1/xc/doc/hardcopy/ICCCM/icccm.PS.Z}

Scheifler, Robert W., ``X Window System Protocol,''
{\tt ftp://ftp.x.org/pub/R6.1/xc/doc/hardcopy/XProtocol/proto.PS.Z}

Treese, G. Winfield and Wolman, Alec, ``X Through the Firewall, and
Other Application Relays,'' Digital Equipment Corporation Cambridge
Research Lab, Technical Report Series, CRL 93/10, May 3, 1993.

\end{document}
